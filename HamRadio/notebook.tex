
% Default to the notebook output style

    


% Inherit from the specified cell style.




    
\documentclass[11pt]{article}

    
    
    \usepackage[T1]{fontenc}
    % Nicer default font (+ math font) than Computer Modern for most use cases
    \usepackage{mathpazo}

    % Basic figure setup, for now with no caption control since it's done
    % automatically by Pandoc (which extracts ![](path) syntax from Markdown).
    \usepackage{graphicx}
    % We will generate all images so they have a width \maxwidth. This means
    % that they will get their normal width if they fit onto the page, but
    % are scaled down if they would overflow the margins.
    \makeatletter
    \def\maxwidth{\ifdim\Gin@nat@width>\linewidth\linewidth
    \else\Gin@nat@width\fi}
    \makeatother
    \let\Oldincludegraphics\includegraphics
    % Set max figure width to be 80% of text width, for now hardcoded.
    \renewcommand{\includegraphics}[1]{\Oldincludegraphics[width=.8\maxwidth]{#1}}
    % Ensure that by default, figures have no caption (until we provide a
    % proper Figure object with a Caption API and a way to capture that
    % in the conversion process - todo).
    \usepackage{caption}
    \DeclareCaptionLabelFormat{nolabel}{}
    \captionsetup{labelformat=nolabel}

    \usepackage{adjustbox} % Used to constrain images to a maximum size 
    \usepackage{xcolor} % Allow colors to be defined
    \usepackage{enumerate} % Needed for markdown enumerations to work
    \usepackage{geometry} % Used to adjust the document margins
    \usepackage{amsmath} % Equations
    \usepackage{amssymb} % Equations
    \usepackage{textcomp} % defines textquotesingle
    % Hack from http://tex.stackexchange.com/a/47451/13684:
    \AtBeginDocument{%
        \def\PYZsq{\textquotesingle}% Upright quotes in Pygmentized code
    }
    \usepackage{upquote} % Upright quotes for verbatim code
    \usepackage{eurosym} % defines \euro
    \usepackage[mathletters]{ucs} % Extended unicode (utf-8) support
    \usepackage[utf8x]{inputenc} % Allow utf-8 characters in the tex document
    \usepackage{fancyvrb} % verbatim replacement that allows latex
    \usepackage{grffile} % extends the file name processing of package graphics 
                         % to support a larger range 
    % The hyperref package gives us a pdf with properly built
    % internal navigation ('pdf bookmarks' for the table of contents,
    % internal cross-reference links, web links for URLs, etc.)
    \usepackage{hyperref}
    \usepackage{longtable} % longtable support required by pandoc >1.10
    \usepackage{booktabs}  % table support for pandoc > 1.12.2
    \usepackage[inline]{enumitem} % IRkernel/repr support (it uses the enumerate* environment)
    \usepackage[normalem]{ulem} % ulem is needed to support strikethroughs (\sout)
                                % normalem makes italics be italics, not underlines
    

    
    
    % Colors for the hyperref package
    \definecolor{urlcolor}{rgb}{0,.145,.698}
    \definecolor{linkcolor}{rgb}{.71,0.21,0.01}
    \definecolor{citecolor}{rgb}{.12,.54,.11}

    % ANSI colors
    \definecolor{ansi-black}{HTML}{3E424D}
    \definecolor{ansi-black-intense}{HTML}{282C36}
    \definecolor{ansi-red}{HTML}{E75C58}
    \definecolor{ansi-red-intense}{HTML}{B22B31}
    \definecolor{ansi-green}{HTML}{00A250}
    \definecolor{ansi-green-intense}{HTML}{007427}
    \definecolor{ansi-yellow}{HTML}{DDB62B}
    \definecolor{ansi-yellow-intense}{HTML}{B27D12}
    \definecolor{ansi-blue}{HTML}{208FFB}
    \definecolor{ansi-blue-intense}{HTML}{0065CA}
    \definecolor{ansi-magenta}{HTML}{D160C4}
    \definecolor{ansi-magenta-intense}{HTML}{A03196}
    \definecolor{ansi-cyan}{HTML}{60C6C8}
    \definecolor{ansi-cyan-intense}{HTML}{258F8F}
    \definecolor{ansi-white}{HTML}{C5C1B4}
    \definecolor{ansi-white-intense}{HTML}{A1A6B2}

    % commands and environments needed by pandoc snippets
    % extracted from the output of `pandoc -s`
    \providecommand{\tightlist}{%
      \setlength{\itemsep}{0pt}\setlength{\parskip}{0pt}}
    \DefineVerbatimEnvironment{Highlighting}{Verbatim}{commandchars=\\\{\}}
    % Add ',fontsize=\small' for more characters per line
    \newenvironment{Shaded}{}{}
    \newcommand{\KeywordTok}[1]{\textcolor[rgb]{0.00,0.44,0.13}{\textbf{{#1}}}}
    \newcommand{\DataTypeTok}[1]{\textcolor[rgb]{0.56,0.13,0.00}{{#1}}}
    \newcommand{\DecValTok}[1]{\textcolor[rgb]{0.25,0.63,0.44}{{#1}}}
    \newcommand{\BaseNTok}[1]{\textcolor[rgb]{0.25,0.63,0.44}{{#1}}}
    \newcommand{\FloatTok}[1]{\textcolor[rgb]{0.25,0.63,0.44}{{#1}}}
    \newcommand{\CharTok}[1]{\textcolor[rgb]{0.25,0.44,0.63}{{#1}}}
    \newcommand{\StringTok}[1]{\textcolor[rgb]{0.25,0.44,0.63}{{#1}}}
    \newcommand{\CommentTok}[1]{\textcolor[rgb]{0.38,0.63,0.69}{\textit{{#1}}}}
    \newcommand{\OtherTok}[1]{\textcolor[rgb]{0.00,0.44,0.13}{{#1}}}
    \newcommand{\AlertTok}[1]{\textcolor[rgb]{1.00,0.00,0.00}{\textbf{{#1}}}}
    \newcommand{\FunctionTok}[1]{\textcolor[rgb]{0.02,0.16,0.49}{{#1}}}
    \newcommand{\RegionMarkerTok}[1]{{#1}}
    \newcommand{\ErrorTok}[1]{\textcolor[rgb]{1.00,0.00,0.00}{\textbf{{#1}}}}
    \newcommand{\NormalTok}[1]{{#1}}
    
    % Additional commands for more recent versions of Pandoc
    \newcommand{\ConstantTok}[1]{\textcolor[rgb]{0.53,0.00,0.00}{{#1}}}
    \newcommand{\SpecialCharTok}[1]{\textcolor[rgb]{0.25,0.44,0.63}{{#1}}}
    \newcommand{\VerbatimStringTok}[1]{\textcolor[rgb]{0.25,0.44,0.63}{{#1}}}
    \newcommand{\SpecialStringTok}[1]{\textcolor[rgb]{0.73,0.40,0.53}{{#1}}}
    \newcommand{\ImportTok}[1]{{#1}}
    \newcommand{\DocumentationTok}[1]{\textcolor[rgb]{0.73,0.13,0.13}{\textit{{#1}}}}
    \newcommand{\AnnotationTok}[1]{\textcolor[rgb]{0.38,0.63,0.69}{\textbf{\textit{{#1}}}}}
    \newcommand{\CommentVarTok}[1]{\textcolor[rgb]{0.38,0.63,0.69}{\textbf{\textit{{#1}}}}}
    \newcommand{\VariableTok}[1]{\textcolor[rgb]{0.10,0.09,0.49}{{#1}}}
    \newcommand{\ControlFlowTok}[1]{\textcolor[rgb]{0.00,0.44,0.13}{\textbf{{#1}}}}
    \newcommand{\OperatorTok}[1]{\textcolor[rgb]{0.40,0.40,0.40}{{#1}}}
    \newcommand{\BuiltInTok}[1]{{#1}}
    \newcommand{\ExtensionTok}[1]{{#1}}
    \newcommand{\PreprocessorTok}[1]{\textcolor[rgb]{0.74,0.48,0.00}{{#1}}}
    \newcommand{\AttributeTok}[1]{\textcolor[rgb]{0.49,0.56,0.16}{{#1}}}
    \newcommand{\InformationTok}[1]{\textcolor[rgb]{0.38,0.63,0.69}{\textbf{\textit{{#1}}}}}
    \newcommand{\WarningTok}[1]{\textcolor[rgb]{0.38,0.63,0.69}{\textbf{\textit{{#1}}}}}
    
    
    % Define a nice break command that doesn't care if a line doesn't already
    % exist.
    \def\br{\hspace*{\fill} \\* }
    % Math Jax compatability definitions
    \def\gt{>}
    \def\lt{<}
    % Document parameters
    \title{TLine3}
    
    
    

    % Pygments definitions
    
\makeatletter
\def\PY@reset{\let\PY@it=\relax \let\PY@bf=\relax%
    \let\PY@ul=\relax \let\PY@tc=\relax%
    \let\PY@bc=\relax \let\PY@ff=\relax}
\def\PY@tok#1{\csname PY@tok@#1\endcsname}
\def\PY@toks#1+{\ifx\relax#1\empty\else%
    \PY@tok{#1}\expandafter\PY@toks\fi}
\def\PY@do#1{\PY@bc{\PY@tc{\PY@ul{%
    \PY@it{\PY@bf{\PY@ff{#1}}}}}}}
\def\PY#1#2{\PY@reset\PY@toks#1+\relax+\PY@do{#2}}

\expandafter\def\csname PY@tok@w\endcsname{\def\PY@tc##1{\textcolor[rgb]{0.73,0.73,0.73}{##1}}}
\expandafter\def\csname PY@tok@c\endcsname{\let\PY@it=\textit\def\PY@tc##1{\textcolor[rgb]{0.25,0.50,0.50}{##1}}}
\expandafter\def\csname PY@tok@cp\endcsname{\def\PY@tc##1{\textcolor[rgb]{0.74,0.48,0.00}{##1}}}
\expandafter\def\csname PY@tok@k\endcsname{\let\PY@bf=\textbf\def\PY@tc##1{\textcolor[rgb]{0.00,0.50,0.00}{##1}}}
\expandafter\def\csname PY@tok@kp\endcsname{\def\PY@tc##1{\textcolor[rgb]{0.00,0.50,0.00}{##1}}}
\expandafter\def\csname PY@tok@kt\endcsname{\def\PY@tc##1{\textcolor[rgb]{0.69,0.00,0.25}{##1}}}
\expandafter\def\csname PY@tok@o\endcsname{\def\PY@tc##1{\textcolor[rgb]{0.40,0.40,0.40}{##1}}}
\expandafter\def\csname PY@tok@ow\endcsname{\let\PY@bf=\textbf\def\PY@tc##1{\textcolor[rgb]{0.67,0.13,1.00}{##1}}}
\expandafter\def\csname PY@tok@nb\endcsname{\def\PY@tc##1{\textcolor[rgb]{0.00,0.50,0.00}{##1}}}
\expandafter\def\csname PY@tok@nf\endcsname{\def\PY@tc##1{\textcolor[rgb]{0.00,0.00,1.00}{##1}}}
\expandafter\def\csname PY@tok@nc\endcsname{\let\PY@bf=\textbf\def\PY@tc##1{\textcolor[rgb]{0.00,0.00,1.00}{##1}}}
\expandafter\def\csname PY@tok@nn\endcsname{\let\PY@bf=\textbf\def\PY@tc##1{\textcolor[rgb]{0.00,0.00,1.00}{##1}}}
\expandafter\def\csname PY@tok@ne\endcsname{\let\PY@bf=\textbf\def\PY@tc##1{\textcolor[rgb]{0.82,0.25,0.23}{##1}}}
\expandafter\def\csname PY@tok@nv\endcsname{\def\PY@tc##1{\textcolor[rgb]{0.10,0.09,0.49}{##1}}}
\expandafter\def\csname PY@tok@no\endcsname{\def\PY@tc##1{\textcolor[rgb]{0.53,0.00,0.00}{##1}}}
\expandafter\def\csname PY@tok@nl\endcsname{\def\PY@tc##1{\textcolor[rgb]{0.63,0.63,0.00}{##1}}}
\expandafter\def\csname PY@tok@ni\endcsname{\let\PY@bf=\textbf\def\PY@tc##1{\textcolor[rgb]{0.60,0.60,0.60}{##1}}}
\expandafter\def\csname PY@tok@na\endcsname{\def\PY@tc##1{\textcolor[rgb]{0.49,0.56,0.16}{##1}}}
\expandafter\def\csname PY@tok@nt\endcsname{\let\PY@bf=\textbf\def\PY@tc##1{\textcolor[rgb]{0.00,0.50,0.00}{##1}}}
\expandafter\def\csname PY@tok@nd\endcsname{\def\PY@tc##1{\textcolor[rgb]{0.67,0.13,1.00}{##1}}}
\expandafter\def\csname PY@tok@s\endcsname{\def\PY@tc##1{\textcolor[rgb]{0.73,0.13,0.13}{##1}}}
\expandafter\def\csname PY@tok@sd\endcsname{\let\PY@it=\textit\def\PY@tc##1{\textcolor[rgb]{0.73,0.13,0.13}{##1}}}
\expandafter\def\csname PY@tok@si\endcsname{\let\PY@bf=\textbf\def\PY@tc##1{\textcolor[rgb]{0.73,0.40,0.53}{##1}}}
\expandafter\def\csname PY@tok@se\endcsname{\let\PY@bf=\textbf\def\PY@tc##1{\textcolor[rgb]{0.73,0.40,0.13}{##1}}}
\expandafter\def\csname PY@tok@sr\endcsname{\def\PY@tc##1{\textcolor[rgb]{0.73,0.40,0.53}{##1}}}
\expandafter\def\csname PY@tok@ss\endcsname{\def\PY@tc##1{\textcolor[rgb]{0.10,0.09,0.49}{##1}}}
\expandafter\def\csname PY@tok@sx\endcsname{\def\PY@tc##1{\textcolor[rgb]{0.00,0.50,0.00}{##1}}}
\expandafter\def\csname PY@tok@m\endcsname{\def\PY@tc##1{\textcolor[rgb]{0.40,0.40,0.40}{##1}}}
\expandafter\def\csname PY@tok@gh\endcsname{\let\PY@bf=\textbf\def\PY@tc##1{\textcolor[rgb]{0.00,0.00,0.50}{##1}}}
\expandafter\def\csname PY@tok@gu\endcsname{\let\PY@bf=\textbf\def\PY@tc##1{\textcolor[rgb]{0.50,0.00,0.50}{##1}}}
\expandafter\def\csname PY@tok@gd\endcsname{\def\PY@tc##1{\textcolor[rgb]{0.63,0.00,0.00}{##1}}}
\expandafter\def\csname PY@tok@gi\endcsname{\def\PY@tc##1{\textcolor[rgb]{0.00,0.63,0.00}{##1}}}
\expandafter\def\csname PY@tok@gr\endcsname{\def\PY@tc##1{\textcolor[rgb]{1.00,0.00,0.00}{##1}}}
\expandafter\def\csname PY@tok@ge\endcsname{\let\PY@it=\textit}
\expandafter\def\csname PY@tok@gs\endcsname{\let\PY@bf=\textbf}
\expandafter\def\csname PY@tok@gp\endcsname{\let\PY@bf=\textbf\def\PY@tc##1{\textcolor[rgb]{0.00,0.00,0.50}{##1}}}
\expandafter\def\csname PY@tok@go\endcsname{\def\PY@tc##1{\textcolor[rgb]{0.53,0.53,0.53}{##1}}}
\expandafter\def\csname PY@tok@gt\endcsname{\def\PY@tc##1{\textcolor[rgb]{0.00,0.27,0.87}{##1}}}
\expandafter\def\csname PY@tok@err\endcsname{\def\PY@bc##1{\setlength{\fboxsep}{0pt}\fcolorbox[rgb]{1.00,0.00,0.00}{1,1,1}{\strut ##1}}}
\expandafter\def\csname PY@tok@kc\endcsname{\let\PY@bf=\textbf\def\PY@tc##1{\textcolor[rgb]{0.00,0.50,0.00}{##1}}}
\expandafter\def\csname PY@tok@kd\endcsname{\let\PY@bf=\textbf\def\PY@tc##1{\textcolor[rgb]{0.00,0.50,0.00}{##1}}}
\expandafter\def\csname PY@tok@kn\endcsname{\let\PY@bf=\textbf\def\PY@tc##1{\textcolor[rgb]{0.00,0.50,0.00}{##1}}}
\expandafter\def\csname PY@tok@kr\endcsname{\let\PY@bf=\textbf\def\PY@tc##1{\textcolor[rgb]{0.00,0.50,0.00}{##1}}}
\expandafter\def\csname PY@tok@bp\endcsname{\def\PY@tc##1{\textcolor[rgb]{0.00,0.50,0.00}{##1}}}
\expandafter\def\csname PY@tok@fm\endcsname{\def\PY@tc##1{\textcolor[rgb]{0.00,0.00,1.00}{##1}}}
\expandafter\def\csname PY@tok@vc\endcsname{\def\PY@tc##1{\textcolor[rgb]{0.10,0.09,0.49}{##1}}}
\expandafter\def\csname PY@tok@vg\endcsname{\def\PY@tc##1{\textcolor[rgb]{0.10,0.09,0.49}{##1}}}
\expandafter\def\csname PY@tok@vi\endcsname{\def\PY@tc##1{\textcolor[rgb]{0.10,0.09,0.49}{##1}}}
\expandafter\def\csname PY@tok@vm\endcsname{\def\PY@tc##1{\textcolor[rgb]{0.10,0.09,0.49}{##1}}}
\expandafter\def\csname PY@tok@sa\endcsname{\def\PY@tc##1{\textcolor[rgb]{0.73,0.13,0.13}{##1}}}
\expandafter\def\csname PY@tok@sb\endcsname{\def\PY@tc##1{\textcolor[rgb]{0.73,0.13,0.13}{##1}}}
\expandafter\def\csname PY@tok@sc\endcsname{\def\PY@tc##1{\textcolor[rgb]{0.73,0.13,0.13}{##1}}}
\expandafter\def\csname PY@tok@dl\endcsname{\def\PY@tc##1{\textcolor[rgb]{0.73,0.13,0.13}{##1}}}
\expandafter\def\csname PY@tok@s2\endcsname{\def\PY@tc##1{\textcolor[rgb]{0.73,0.13,0.13}{##1}}}
\expandafter\def\csname PY@tok@sh\endcsname{\def\PY@tc##1{\textcolor[rgb]{0.73,0.13,0.13}{##1}}}
\expandafter\def\csname PY@tok@s1\endcsname{\def\PY@tc##1{\textcolor[rgb]{0.73,0.13,0.13}{##1}}}
\expandafter\def\csname PY@tok@mb\endcsname{\def\PY@tc##1{\textcolor[rgb]{0.40,0.40,0.40}{##1}}}
\expandafter\def\csname PY@tok@mf\endcsname{\def\PY@tc##1{\textcolor[rgb]{0.40,0.40,0.40}{##1}}}
\expandafter\def\csname PY@tok@mh\endcsname{\def\PY@tc##1{\textcolor[rgb]{0.40,0.40,0.40}{##1}}}
\expandafter\def\csname PY@tok@mi\endcsname{\def\PY@tc##1{\textcolor[rgb]{0.40,0.40,0.40}{##1}}}
\expandafter\def\csname PY@tok@il\endcsname{\def\PY@tc##1{\textcolor[rgb]{0.40,0.40,0.40}{##1}}}
\expandafter\def\csname PY@tok@mo\endcsname{\def\PY@tc##1{\textcolor[rgb]{0.40,0.40,0.40}{##1}}}
\expandafter\def\csname PY@tok@ch\endcsname{\let\PY@it=\textit\def\PY@tc##1{\textcolor[rgb]{0.25,0.50,0.50}{##1}}}
\expandafter\def\csname PY@tok@cm\endcsname{\let\PY@it=\textit\def\PY@tc##1{\textcolor[rgb]{0.25,0.50,0.50}{##1}}}
\expandafter\def\csname PY@tok@cpf\endcsname{\let\PY@it=\textit\def\PY@tc##1{\textcolor[rgb]{0.25,0.50,0.50}{##1}}}
\expandafter\def\csname PY@tok@c1\endcsname{\let\PY@it=\textit\def\PY@tc##1{\textcolor[rgb]{0.25,0.50,0.50}{##1}}}
\expandafter\def\csname PY@tok@cs\endcsname{\let\PY@it=\textit\def\PY@tc##1{\textcolor[rgb]{0.25,0.50,0.50}{##1}}}

\def\PYZbs{\char`\\}
\def\PYZus{\char`\_}
\def\PYZob{\char`\{}
\def\PYZcb{\char`\}}
\def\PYZca{\char`\^}
\def\PYZam{\char`\&}
\def\PYZlt{\char`\<}
\def\PYZgt{\char`\>}
\def\PYZsh{\char`\#}
\def\PYZpc{\char`\%}
\def\PYZdl{\char`\$}
\def\PYZhy{\char`\-}
\def\PYZsq{\char`\'}
\def\PYZdq{\char`\"}
\def\PYZti{\char`\~}
% for compatibility with earlier versions
\def\PYZat{@}
\def\PYZlb{[}
\def\PYZrb{]}
\makeatother


    % Exact colors from NB
    \definecolor{incolor}{rgb}{0.0, 0.0, 0.5}
    \definecolor{outcolor}{rgb}{0.545, 0.0, 0.0}



    
    % Prevent overflowing lines due to hard-to-break entities
    \sloppy 
    % Setup hyperref package
    \hypersetup{
      breaklinks=true,  % so long urls are correctly broken across lines
      colorlinks=true,
      urlcolor=urlcolor,
      linkcolor=linkcolor,
      citecolor=citecolor,
      }
    % Slightly bigger margins than the latex defaults
    
    \geometry{verbose,tmargin=1in,bmargin=1in,lmargin=1in,rmargin=1in}
    
    

    \begin{document}
    
    
    \maketitle
    
    

    
    \begin{Verbatim}[commandchars=\\\{\}]
{\color{incolor}In [{\color{incolor}2}]:} \PY{k+kn}{import} \PY{n+nn}{numpy} \PY{k}{as} \PY{n+nn}{np}
        \PY{k+kn}{import} \PY{n+nn}{matplotlib}\PY{n+nn}{.}\PY{n+nn}{pyplot} \PY{k}{as} \PY{n+nn}{plt}
        \PY{k+kn}{import} \PY{n+nn}{skrf} \PY{k}{as} \PY{n+nn}{rf}
        \PY{k+kn}{from} \PY{n+nn}{skrf} \PY{k}{import} \PY{n}{Network}
        \PY{k+kn}{from} \PY{n+nn}{skrf} \PY{k}{import} \PY{n}{plotting}
        \PY{k+kn}{import} \PY{n+nn}{math}
        \PY{c+c1}{\PYZsh{} The example uses pylab \PYZhy{} I think this further encourages sloppy usage. }
        \PY{c+c1}{\PYZsh{} from pylab import *}
\end{Verbatim}


    Modified By Reason 17-Jan-22 CBL Original Online docs:

https://scikit-rf.readthedocs.io/en/latest/tutorials/Plotting.html

github associated with this: https://github.com/scikit-rf/

https://scikit-rf.readthedocs.io/en/latest/examples/networktheory/Transmission\%20Line\%20Properties\%20and\%20Manipulations.html

Voltages and Currents

Now assume that the previous circuit is excited by a source delivering a
voltage of 1 Volt and associated to a source impedance of \$Z\_s = 100
\Omega \$ 

    \begin{Verbatim}[commandchars=\\\{\}]
{\color{incolor}In [{\color{incolor}3}]:} \PY{c+c1}{\PYZsh{} skrf figure styling}
        \PY{n}{rf}\PY{o}{.}\PY{n}{stylely}\PY{p}{(}\PY{p}{)}
\end{Verbatim}


    \begin{Verbatim}[commandchars=\\\{\}]
{\color{incolor}In [{\color{incolor}4}]:} \PY{n}{Z\PYZus{}s} \PY{o}{=} \PY{l+m+mi}{100}  \PY{c+c1}{\PYZsh{} Ohm}
        \PY{n}{V\PYZus{}s} \PY{o}{=} \PY{l+m+mi}{1}    \PY{c+c1}{\PYZsh{} V}
        
        \PY{c+c1}{\PYZsh{} From previous example}
        \PY{n}{freq} \PY{o}{=} \PY{n}{rf}\PY{o}{.}\PY{n}{Frequency}\PY{p}{(}\PY{l+m+mi}{250}\PY{p}{,} \PY{n}{npoints}\PY{o}{=}\PY{l+m+mi}{1}\PY{p}{,} \PY{n}{unit}\PY{o}{=}\PY{l+s+s1}{\PYZsq{}}\PY{l+s+s1}{MHz}\PY{l+s+s1}{\PYZsq{}}\PY{p}{)}
        \PY{n}{Z\PYZus{}0} \PY{o}{=} \PY{l+m+mi}{75}  \PY{c+c1}{\PYZsh{} Ohm}
        \PY{n}{Z\PYZus{}L} \PY{o}{=} \PY{l+m+mi}{150}  \PY{c+c1}{\PYZsh{} Ohm}
        \PY{n}{d}   \PY{o}{=} \PY{l+m+mi}{12}  \PY{c+c1}{\PYZsh{} m}
        \PY{n}{VF}  \PY{o}{=} \PY{l+m+mf}{0.67}
        \PY{n}{att} \PY{o}{=} \PY{l+m+mf}{0.02} \PY{c+c1}{\PYZsh{} Np/m. Equivalent to 0.1737 dB/m}
        \PY{n}{alpha} \PY{o}{=} \PY{n}{att}  \PY{c+c1}{\PYZsh{} Np/m !}
        \PY{n}{beta} \PY{o}{=} \PY{n}{freq}\PY{o}{.}\PY{n}{w}\PY{o}{/}\PY{n}{rf}\PY{o}{.}\PY{n}{c}\PY{o}{/}\PY{n}{VF}
        \PY{n}{gamma} \PY{o}{=} \PY{n}{alpha} \PY{o}{+} \PY{l+m+mi}{1}\PY{n}{j}\PY{o}{*}\PY{n}{beta}
        \PY{n}{Gamma\PYZus{}in} \PY{o}{=} \PY{n}{rf}\PY{o}{.}\PY{n}{zl\PYZus{}2\PYZus{}Gamma\PYZus{}in}\PY{p}{(}\PY{n}{Z\PYZus{}0}\PY{p}{,} \PY{n}{Z\PYZus{}L}\PY{p}{,} \PY{n}{theta}\PY{o}{=}\PY{n}{gamma}\PY{o}{*}\PY{n}{d}\PY{p}{)}
\end{Verbatim}


    \begin{Verbatim}[commandchars=\\\{\}]
{\color{incolor}In [{\color{incolor}5}]:} \PY{c+c1}{\PYZsh{} Z in is not yet defined}
        \PY{n}{Z\PYZus{}in} \PY{o}{=} \PY{n}{rf}\PY{o}{.}\PY{n}{zl\PYZus{}2\PYZus{}zin}\PY{p}{(}\PY{n}{Z\PYZus{}0}\PY{p}{,} \PY{n}{Z\PYZus{}L}\PY{p}{,} \PY{n}{gamma} \PY{o}{*} \PY{n}{d}\PY{p}{)}
        \PY{n+nb}{print}\PY{p}{(}\PY{l+s+s1}{\PYZsq{}}\PY{l+s+s1}{Input impedance Z\PYZus{}in=}\PY{l+s+s1}{\PYZsq{}}\PY{p}{,} \PY{n}{Z\PYZus{}in}\PY{p}{)}
\end{Verbatim}


    \begin{Verbatim}[commandchars=\\\{\}]
Input impedance Z\_in= [94.79804694+29.52482706j]

    \end{Verbatim}

    At the input of the transmission line, the voltage is a voltage divider
circuit:

\$V\_\{in\} = V\_\{s\} \{Z\_\{in\} \over{Z_s + Z_{in}}\} \$

    \begin{Verbatim}[commandchars=\\\{\}]
{\color{incolor}In [{\color{incolor}6}]:} \PY{n}{V\PYZus{}in} \PY{o}{=} \PY{n}{V\PYZus{}s} \PY{o}{*} \PY{n}{Z\PYZus{}in} \PY{o}{/} \PY{p}{(}\PY{n}{Z\PYZus{}s} \PY{o}{+} \PY{n}{Z\PYZus{}in}\PY{p}{)}
        \PY{n+nb}{print}\PY{p}{(}\PY{l+s+s1}{\PYZsq{}}\PY{l+s+s1}{Voltage at transmission line input : V\PYZus{}in = }\PY{l+s+s1}{\PYZsq{}}\PY{p}{,} \PY{n}{V\PYZus{}in}\PY{p}{,} \PY{l+s+s1}{\PYZsq{}}\PY{l+s+s1}{ V}\PY{l+s+s1}{\PYZsq{}}\PY{p}{)}
\end{Verbatim}


    \begin{Verbatim}[commandchars=\\\{\}]
Voltage at transmission line input : V\_in =  [0.49817591+0.07605964j]  V

    \end{Verbatim}

    and the current at the input of the transmission line is:

\$I\_\{in\} = \{V\_\{s\}\over{Z_s + Z_{in}}\} \$

    \begin{Verbatim}[commandchars=\\\{\}]
{\color{incolor}In [{\color{incolor}7}]:} \PY{n}{I\PYZus{}in} \PY{o}{=} \PY{n}{V\PYZus{}s} \PY{o}{/} \PY{p}{(}\PY{n}{Z\PYZus{}s} \PY{o}{+} \PY{n}{Z\PYZus{}in}\PY{p}{)}
        \PY{n+nb}{print}\PY{p}{(}\PY{l+s+s1}{\PYZsq{}}\PY{l+s+s1}{Current at transmission line input : I\PYZus{}in = }\PY{l+s+s1}{\PYZsq{}}\PY{p}{,} \PY{n}{I\PYZus{}in}\PY{p}{,} \PY{l+s+s1}{\PYZsq{}}\PY{l+s+s1}{ A}\PY{l+s+s1}{\PYZsq{}}\PY{p}{)}
\end{Verbatim}


    \begin{Verbatim}[commandchars=\\\{\}]
Current at transmission line input : I\_in =  [0.00501824-0.0007606j]  A

    \end{Verbatim}

    which represent a power of:

\$P\_\{in\} = \{1\over 2\} R {[}V\_\{in\}I\^{}*\_\{in\}{]} \$

    \begin{Verbatim}[commandchars=\\\{\}]
{\color{incolor}In [{\color{incolor}8}]:} \PY{n}{P\PYZus{}in} \PY{o}{=} \PY{l+m+mi}{1}\PY{o}{/}\PY{l+m+mi}{2} \PY{o}{*} \PY{n}{np}\PY{o}{.}\PY{n}{real}\PY{p}{(}\PY{n}{V\PYZus{}in} \PY{o}{*} \PY{n}{np}\PY{o}{.}\PY{n}{conjugate}\PY{p}{(}\PY{n}{I\PYZus{}in}\PY{p}{)}\PY{p}{)}
        \PY{n+nb}{print}\PY{p}{(}\PY{l+s+s1}{\PYZsq{}}\PY{l+s+s1}{Input Power : P\PYZus{}in = }\PY{l+s+s1}{\PYZsq{}}\PY{p}{,} \PY{n}{P\PYZus{}in}\PY{p}{,} \PY{l+s+s1}{\PYZsq{}}\PY{l+s+s1}{W}\PY{l+s+s1}{\PYZsq{}}\PY{p}{)}
\end{Verbatim}


    \begin{Verbatim}[commandchars=\\\{\}]
Input Power : P\_in =  [0.00122106] W

    \end{Verbatim}

    The reflected power is:

\$P\_r = \textbar{}\Gamma\emph{\{in\}\textbar{}\^{}2 P}\{in\} \$

    \begin{Verbatim}[commandchars=\\\{\}]
{\color{incolor}In [{\color{incolor}9}]:} \PY{n}{P\PYZus{}r} \PY{o}{=} \PY{n}{np}\PY{o}{.}\PY{n}{abs}\PY{p}{(}\PY{n}{Gamma\PYZus{}in}\PY{p}{)}\PY{o}{*}\PY{o}{*}\PY{l+m+mi}{2} \PY{o}{*} \PY{n}{P\PYZus{}in}
        \PY{n+nb}{print}\PY{p}{(}\PY{l+s+s1}{\PYZsq{}}\PY{l+s+s1}{Reflected power : P\PYZus{}r = }\PY{l+s+s1}{\PYZsq{}}\PY{p}{,} \PY{n}{P\PYZus{}r}\PY{p}{,} \PY{l+s+s1}{\PYZsq{}}\PY{l+s+s1}{W}\PY{l+s+s1}{\PYZsq{}}\PY{p}{)}
\end{Verbatim}


    \begin{Verbatim}[commandchars=\\\{\}]
Reflected power : P\_r =  [5.19482699e-05] W

    \end{Verbatim}

    The voltage and current at the load can be deduced from the ABCD
parameters of the line of length L:

    \begin{Verbatim}[commandchars=\\\{\}]
{\color{incolor}In [{\color{incolor}10}]:} \PY{n}{V\PYZus{}out}\PY{p}{,} \PY{n}{I\PYZus{}out} \PY{o}{=} \PY{n}{rf}\PY{o}{.}\PY{n}{voltage\PYZus{}current\PYZus{}propagation}\PY{p}{(}\PY{n}{V\PYZus{}in}\PY{p}{,} \PY{n}{I\PYZus{}in}\PY{p}{,} \PY{n}{Z\PYZus{}0}\PY{p}{,}\PY{n}{theta}\PY{o}{=} \PY{n}{gamma}\PY{o}{*}\PY{n}{d}\PY{p}{)}
         \PY{n+nb}{print}\PY{p}{(}\PY{l+s+s1}{\PYZsq{}}\PY{l+s+s1}{Voltage at load: V\PYZus{}out = }\PY{l+s+s1}{\PYZsq{}}\PY{p}{,} \PY{n}{V\PYZus{}out}\PY{p}{,} \PY{l+s+s1}{\PYZsq{}}\PY{l+s+s1}{V}\PY{l+s+s1}{\PYZsq{}}\PY{p}{)}
         \PY{n+nb}{print}\PY{p}{(}\PY{l+s+s1}{\PYZsq{}}\PY{l+s+s1}{Current at load: I\PYZus{}out = }\PY{l+s+s1}{\PYZsq{}}\PY{p}{,} \PY{n}{I\PYZus{}out}\PY{p}{,} \PY{l+s+s1}{\PYZsq{}}\PY{l+s+s1}{A}\PY{l+s+s1}{\PYZsq{}}\PY{p}{)}
\end{Verbatim}


    \begin{Verbatim}[commandchars=\\\{\}]
Voltage at load: V\_out =  [0.41779105+0.18944365j] V
Current at load: I\_out =  [0.00278527+0.00126296j] A

    \end{Verbatim}

    Note that voltages and currents are expressed a peak values. RMS values
are:

    \begin{Verbatim}[commandchars=\\\{\}]
{\color{incolor}In [{\color{incolor}11}]:} \PY{n+nb}{print}\PY{p}{(}\PY{n+nb}{abs}\PY{p}{(}\PY{n}{V\PYZus{}out}\PY{p}{)}\PY{o}{/}\PY{n}{math}\PY{o}{.}\PY{n}{sqrt}\PY{p}{(}\PY{l+m+mi}{2}\PY{p}{)}\PY{p}{,} \PY{n+nb}{abs}\PY{p}{(}\PY{n}{I\PYZus{}out}\PY{p}{)}\PY{o}{/}\PY{n}{math}\PY{o}{.}\PY{n}{sqrt}\PY{p}{(}\PY{l+m+mi}{2}\PY{p}{)}\PY{p}{)}
\end{Verbatim}


    \begin{Verbatim}[commandchars=\\\{\}]
[0.32437498] [0.0021625]

    \end{Verbatim}

    The power delivered to the load is:

    \begin{Verbatim}[commandchars=\\\{\}]
{\color{incolor}In [{\color{incolor}12}]:} \PY{n}{P\PYZus{}out} \PY{o}{=} \PY{l+m+mi}{1}\PY{o}{/}\PY{l+m+mi}{2} \PY{o}{*} \PY{n}{np}\PY{o}{.}\PY{n}{real}\PY{p}{(}\PY{n}{V\PYZus{}out} \PY{o}{*} \PY{n}{np}\PY{o}{.}\PY{n}{conjugate}\PY{p}{(}\PY{n}{I\PYZus{}out}\PY{p}{)}\PY{p}{)}
         \PY{n+nb}{print}\PY{p}{(}\PY{l+s+s1}{\PYZsq{}}\PY{l+s+s1}{Power delivered to the load : P\PYZus{}out = }\PY{l+s+s1}{\PYZsq{}}\PY{p}{,} \PY{n}{P\PYZus{}out}\PY{p}{,} \PY{l+s+s1}{\PYZsq{}}\PY{l+s+s1}{ W}\PY{l+s+s1}{\PYZsq{}}\PY{p}{)}
\end{Verbatim}


    \begin{Verbatim}[commandchars=\\\{\}]
Power delivered to the load : P\_out =  [0.00070146]  W

    \end{Verbatim}

    Voltage and current are plotted below against the transmission line
length (pay attention to the sign of d in the voltage and current
propagation: as we go from source ( \(z=d\)) to the load (\(z=0\)),
\$\theta \$ goes in the opposite direction and should be inversed)

    \begin{Verbatim}[commandchars=\\\{\}]
{\color{incolor}In [{\color{incolor}13}]:} \PY{n}{ds} \PY{o}{=} \PY{n}{np}\PY{o}{.}\PY{n}{linspace}\PY{p}{(}\PY{l+m+mi}{0}\PY{p}{,} \PY{n}{d}\PY{p}{,} \PY{n}{num}\PY{o}{=}\PY{l+m+mi}{1001}\PY{p}{)}
         
         \PY{n}{thetas} \PY{o}{=} \PY{o}{\PYZhy{}} \PY{n}{gamma}\PY{o}{*}\PY{n}{ds}
         
         \PY{n}{v1} \PY{o}{=} \PY{n}{np}\PY{o}{.}\PY{n}{full\PYZus{}like}\PY{p}{(}\PY{n}{ds}\PY{p}{,} \PY{n}{V\PYZus{}in}\PY{p}{)}
         \PY{n}{i1} \PY{o}{=} \PY{n}{np}\PY{o}{.}\PY{n}{full\PYZus{}like}\PY{p}{(}\PY{n}{ds}\PY{p}{,} \PY{n}{I\PYZus{}in}\PY{p}{)}
         
         \PY{n}{v2}\PY{p}{,} \PY{n}{i2} \PY{o}{=} \PY{n}{rf}\PY{o}{.}\PY{n}{voltage\PYZus{}current\PYZus{}propagation}\PY{p}{(}\PY{n}{v1}\PY{p}{,} \PY{n}{i1}\PY{p}{,} \PY{n}{Z\PYZus{}0}\PY{p}{,} \PY{n}{thetas}\PY{p}{)}
\end{Verbatim}


    \begin{Verbatim}[commandchars=\\\{\}]
<string>:6: ComplexWarning: Casting complex values to real discards the imaginary part

    \end{Verbatim}

    \begin{Verbatim}[commandchars=\\\{\}]
{\color{incolor}In [{\color{incolor}14}]:} \PY{n}{fig}\PY{p}{,} \PY{p}{(}\PY{n}{ax\PYZus{}V}\PY{p}{,} \PY{n}{ax\PYZus{}I}\PY{p}{)} \PY{o}{=} \PY{n}{plt}\PY{o}{.}\PY{n}{subplots}\PY{p}{(}\PY{l+m+mi}{2}\PY{p}{,} \PY{l+m+mi}{1}\PY{p}{,} \PY{n}{sharex}\PY{o}{=}\PY{k+kc}{True}\PY{p}{)}
         \PY{n}{ax\PYZus{}V}\PY{o}{.}\PY{n}{plot}\PY{p}{(}\PY{n}{ds}\PY{p}{,} \PY{n+nb}{abs}\PY{p}{(}\PY{n}{v2}\PY{p}{)}\PY{p}{,} \PY{n}{lw}\PY{o}{=}\PY{l+m+mi}{2}\PY{p}{)}
         \PY{n}{ax\PYZus{}I}\PY{o}{.}\PY{n}{plot}\PY{p}{(}\PY{n}{ds}\PY{p}{,} \PY{n+nb}{abs}\PY{p}{(}\PY{n}{i2}\PY{p}{)}\PY{p}{,} \PY{n}{lw}\PY{o}{=}\PY{l+m+mi}{2}\PY{p}{,} \PY{n}{c}\PY{o}{=}\PY{l+s+s1}{\PYZsq{}}\PY{l+s+s1}{C1}\PY{l+s+s1}{\PYZsq{}}\PY{p}{)}
         \PY{n}{ax\PYZus{}I}\PY{o}{.}\PY{n}{set\PYZus{}xlabel}\PY{p}{(}\PY{l+s+s1}{\PYZsq{}}\PY{l+s+s1}{z [m]}\PY{l+s+s1}{\PYZsq{}}\PY{p}{)}
         \PY{n}{ax\PYZus{}V}\PY{o}{.}\PY{n}{set\PYZus{}ylabel}\PY{p}{(}\PY{l+s+s1}{\PYZsq{}}\PY{l+s+s1}{|V| [V]}\PY{l+s+s1}{\PYZsq{}}\PY{p}{)}
         \PY{n}{ax\PYZus{}I}\PY{o}{.}\PY{n}{set\PYZus{}ylabel}\PY{p}{(}\PY{l+s+s1}{\PYZsq{}}\PY{l+s+s1}{|I| [A]}\PY{l+s+s1}{\PYZsq{}}\PY{p}{)}
         
         
         \PY{n}{ax\PYZus{}V}\PY{o}{.}\PY{n}{axvline}\PY{p}{(}\PY{l+m+mi}{0}\PY{p}{,} \PY{n}{c}\PY{o}{=}\PY{l+s+s1}{\PYZsq{}}\PY{l+s+s1}{k}\PY{l+s+s1}{\PYZsq{}}\PY{p}{,} \PY{n}{lw}\PY{o}{=}\PY{l+m+mi}{5}\PY{p}{)}
         \PY{n}{ax\PYZus{}I}\PY{o}{.}\PY{n}{axvline}\PY{p}{(}\PY{l+m+mi}{0}\PY{p}{,} \PY{n}{c}\PY{o}{=}\PY{l+s+s1}{\PYZsq{}}\PY{l+s+s1}{k}\PY{l+s+s1}{\PYZsq{}}\PY{p}{,} \PY{n}{lw}\PY{o}{=}\PY{l+m+mi}{5}\PY{p}{)}
         \PY{n}{ax\PYZus{}V}\PY{o}{.}\PY{n}{text}\PY{p}{(}\PY{n}{d}\PY{o}{\PYZhy{}}\PY{l+m+mi}{2}\PY{p}{,} \PY{l+m+mf}{0.4}\PY{p}{,} \PY{l+s+s1}{\PYZsq{}}\PY{l+s+s1}{input}\PY{l+s+s1}{\PYZsq{}}\PY{p}{)}
         \PY{n}{ax\PYZus{}V}\PY{o}{.}\PY{n}{text}\PY{p}{(}\PY{l+m+mi}{1}\PY{p}{,} \PY{l+m+mf}{0.6}\PY{p}{,} \PY{l+s+s1}{\PYZsq{}}\PY{l+s+s1}{load}\PY{l+s+s1}{\PYZsq{}}\PY{p}{)}
         \PY{n}{ax\PYZus{}V}\PY{o}{.}\PY{n}{axvline}\PY{p}{(}\PY{n}{d}\PY{p}{,} \PY{n}{c}\PY{o}{=}\PY{l+s+s1}{\PYZsq{}}\PY{l+s+s1}{k}\PY{l+s+s1}{\PYZsq{}}\PY{p}{,} \PY{n}{lw}\PY{o}{=}\PY{l+m+mi}{5}\PY{p}{)}
         \PY{n}{ax\PYZus{}I}\PY{o}{.}\PY{n}{axvline}\PY{p}{(}\PY{n}{d}\PY{p}{,} \PY{n}{c}\PY{o}{=}\PY{l+s+s1}{\PYZsq{}}\PY{l+s+s1}{k}\PY{l+s+s1}{\PYZsq{}}\PY{p}{,} \PY{n}{lw}\PY{o}{=}\PY{l+m+mi}{5}\PY{p}{)}
         
         \PY{n}{ax\PYZus{}I}\PY{o}{.}\PY{n}{set\PYZus{}title}\PY{p}{(}\PY{l+s+s1}{\PYZsq{}}\PY{l+s+s1}{Current}\PY{l+s+s1}{\PYZsq{}}\PY{p}{)}
         \PY{n}{ax\PYZus{}V}\PY{o}{.}\PY{n}{set\PYZus{}title}\PY{p}{(}\PY{l+s+s1}{\PYZsq{}}\PY{l+s+s1}{Voltage}\PY{l+s+s1}{\PYZsq{}}\PY{p}{)}
\end{Verbatim}


\begin{Verbatim}[commandchars=\\\{\}]
{\color{outcolor}Out[{\color{outcolor}14}]:} Text(0.5, 1.0, 'Voltage')
\end{Verbatim}
            
    \begin{center}
    \adjustimage{max size={0.9\linewidth}{0.9\paperheight}}{output_21_1.png}
    \end{center}
    { \hspace*{\fill} \\}
    
    scikit-rf provides media objects too. Next example.

    \begin{Verbatim}[commandchars=\\\{\}]
{\color{incolor}In [{\color{incolor}15}]:} \PY{c+c1}{\PYZsh{}}
         \PY{c+c1}{\PYZsh{} I\PYZsq{}m curious}
         \PY{n}{rf}\PY{o}{.}\PY{n}{zl\PYZus{}2\PYZus{}swr}\PY{p}{(}\PY{n}{Z\PYZus{}0}\PY{p}{,} \PY{n}{Z\PYZus{}L}\PY{p}{)}
\end{Verbatim}


\begin{Verbatim}[commandchars=\\\{\}]
{\color{outcolor}Out[{\color{outcolor}15}]:} array([2.])
\end{Verbatim}
            

    % Add a bibliography block to the postdoc
    
    
    
    \end{document}
